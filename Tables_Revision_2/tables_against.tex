\scriptsize
\begin{tabular}{>{\raggedright}p{0.19\textwidth}|>{\raggedright}p{0.13\textwidth}|>{\raggedright}p{0.5\textwidth}}
\hline 
\textbf{Person/Organization} & \textbf{Jurisdiction} & \textbf{Quote}\tabularnewline
\hline 
Colorado Restaurant Association & Colorado & 99\% of restaurants will likely limit plans for future growth, 98\%
are likely to schedule fewer workers per shift, 92\% are likely to
cut employee hours, 95\% are likely to stop hiring individuals who
need flexibility in scheduling such as students and single parents.\tabularnewline
\hline 
Judy Amabile

(D-Boulder) & Colorado & {[}FWLs may{]} have unintended consequences that, in the end, could
hurt the very people the bill is intended to help, as well as damage
to the restaurant industry, which she said is still facing the effects
of COVID-19, including worker shortages. \tabularnewline
\hline 
Loren Furman

(Colorado Chamber of Commerce) & Colorado & {[}The{]} restrictive scheduling bill was one of the worst bills for
business we\textquoteright ve seen from the Legislature in years,
and that was reflected by the significant backlash from business owners
across the state. \tabularnewline
\hline 
Krissie Harris 

(Evanston Councilmember) & Evanston, IL & I am being bombarded by my residents, by my businesses. They are not
in favor of this. \tabularnewline
\hline 
Carmine Presta 

(Business Owner) & Evanston, IL & The administrative and penalty requirements of this ordinance will
destroy schedule flexibility for the employee.\tabularnewline
\hline 
Evanston Restaurant Association

(50 + Business Owners) & Evanston, IL & This is a misguided proposal that would destroy workplace flexibility
while devastating a broad spectrum of industries that drive Evanston\textquoteright s
economy. We believe our employees value the flexibility and support
we provide for them. This ordinance hurts the very people it is claiming
to help. \tabularnewline
\hline 
Chauncey Rice

(Associate Vice President of Illinois Retail Merchants Association) & Chicago, IL & It\textquoteright s because of policies like this that retailers of
every type and size including pharmacies, grocers, restaurants, and
hardware stores are increasingly unable to keep their doors open. \tabularnewline
\hline 
Brad Tietz 

(Vice President, Chicagoland Chamber of Commerce) & Chicago, IL & Rather than striking a balance that works for both workers and businesses,
this proposal will hinder economic development and employment opportunities
in the communities that need it most. \tabularnewline
\hline 
Yadira Enriquez 

(Grocery Store Owner) & Chicago, IL & All the dedication, determination, persistence and resourcefulness
that helped us succeed is no match for bad policies that threaten
the dreams of entrepreneurs in neighborhoods across Chicago.\tabularnewline
\hline 
Stuart Waldman 

(Valley Industry and Commerce Association) & Los Angeles, CA & This ordinance is a one-size-fits-all approach that was rushed through
without any debate or economic analysis. This is an example of how
not to legislate.\tabularnewline
\hline 
Jessica Duboff 

(Los Angeles Chamber of Commerce) & Los Angeles, CA & Predictive scheduling is often actually restrictive scheduling, imposing
a one-size-fits-all system that threatens the flexibility of employees
and employers.\tabularnewline
\hline 
Shania Roberts

(LA Retail Coordinator and Staffer) & Los Angeles, CA & Employers also need flexible hours and schedules because of how many
employees leave or don\textquoteright t show up at wanted times. But
now it has to have a two week schedule in advance, so if people are
out all the sudden, it will be even more of a scramble.\tabularnewline
\end{tabular}
